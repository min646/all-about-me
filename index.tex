% Options for packages loaded elsewhere
% Options for packages loaded elsewhere
\PassOptionsToPackage{unicode}{hyperref}
\PassOptionsToPackage{hyphens}{url}
\PassOptionsToPackage{dvipsnames,svgnames,x11names}{xcolor}
%
\documentclass[
  letterpaper,
  DIV=11,
  numbers=noendperiod]{scrreprt}
\usepackage{xcolor}
\usepackage{amsmath,amssymb}
\setcounter{secnumdepth}{5}
\usepackage{iftex}
\ifPDFTeX
  \usepackage[T1]{fontenc}
  \usepackage[utf8]{inputenc}
  \usepackage{textcomp} % provide euro and other symbols
\else % if luatex or xetex
  \usepackage{unicode-math} % this also loads fontspec
  \defaultfontfeatures{Scale=MatchLowercase}
  \defaultfontfeatures[\rmfamily]{Ligatures=TeX,Scale=1}
\fi
\usepackage{lmodern}
\ifPDFTeX\else
  % xetex/luatex font selection
\fi
% Use upquote if available, for straight quotes in verbatim environments
\IfFileExists{upquote.sty}{\usepackage{upquote}}{}
\IfFileExists{microtype.sty}{% use microtype if available
  \usepackage[]{microtype}
  \UseMicrotypeSet[protrusion]{basicmath} % disable protrusion for tt fonts
}{}
\makeatletter
\@ifundefined{KOMAClassName}{% if non-KOMA class
  \IfFileExists{parskip.sty}{%
    \usepackage{parskip}
  }{% else
    \setlength{\parindent}{0pt}
    \setlength{\parskip}{6pt plus 2pt minus 1pt}}
}{% if KOMA class
  \KOMAoptions{parskip=half}}
\makeatother
% Make \paragraph and \subparagraph free-standing
\makeatletter
\ifx\paragraph\undefined\else
  \let\oldparagraph\paragraph
  \renewcommand{\paragraph}{
    \@ifstar
      \xxxParagraphStar
      \xxxParagraphNoStar
  }
  \newcommand{\xxxParagraphStar}[1]{\oldparagraph*{#1}\mbox{}}
  \newcommand{\xxxParagraphNoStar}[1]{\oldparagraph{#1}\mbox{}}
\fi
\ifx\subparagraph\undefined\else
  \let\oldsubparagraph\subparagraph
  \renewcommand{\subparagraph}{
    \@ifstar
      \xxxSubParagraphStar
      \xxxSubParagraphNoStar
  }
  \newcommand{\xxxSubParagraphStar}[1]{\oldsubparagraph*{#1}\mbox{}}
  \newcommand{\xxxSubParagraphNoStar}[1]{\oldsubparagraph{#1}\mbox{}}
\fi
\makeatother


\usepackage{longtable,booktabs,array}
\usepackage{calc} % for calculating minipage widths
% Correct order of tables after \paragraph or \subparagraph
\usepackage{etoolbox}
\makeatletter
\patchcmd\longtable{\par}{\if@noskipsec\mbox{}\fi\par}{}{}
\makeatother
% Allow footnotes in longtable head/foot
\IfFileExists{footnotehyper.sty}{\usepackage{footnotehyper}}{\usepackage{footnote}}
\makesavenoteenv{longtable}
\usepackage{graphicx}
\makeatletter
\newsavebox\pandoc@box
\newcommand*\pandocbounded[1]{% scales image to fit in text height/width
  \sbox\pandoc@box{#1}%
  \Gscale@div\@tempa{\textheight}{\dimexpr\ht\pandoc@box+\dp\pandoc@box\relax}%
  \Gscale@div\@tempb{\linewidth}{\wd\pandoc@box}%
  \ifdim\@tempb\p@<\@tempa\p@\let\@tempa\@tempb\fi% select the smaller of both
  \ifdim\@tempa\p@<\p@\scalebox{\@tempa}{\usebox\pandoc@box}%
  \else\usebox{\pandoc@box}%
  \fi%
}
% Set default figure placement to htbp
\def\fps@figure{htbp}
\makeatother





\setlength{\emergencystretch}{3em} % prevent overfull lines

\providecommand{\tightlist}{%
  \setlength{\itemsep}{0pt}\setlength{\parskip}{0pt}}



 


\KOMAoption{captions}{tableheading}
\makeatletter
\@ifpackageloaded{bookmark}{}{\usepackage{bookmark}}
\makeatother
\makeatletter
\@ifpackageloaded{caption}{}{\usepackage{caption}}
\AtBeginDocument{%
\ifdefined\contentsname
  \renewcommand*\contentsname{Table of contents}
\else
  \newcommand\contentsname{Table of contents}
\fi
\ifdefined\listfigurename
  \renewcommand*\listfigurename{List of Figures}
\else
  \newcommand\listfigurename{List of Figures}
\fi
\ifdefined\listtablename
  \renewcommand*\listtablename{List of Tables}
\else
  \newcommand\listtablename{List of Tables}
\fi
\ifdefined\figurename
  \renewcommand*\figurename{Figure}
\else
  \newcommand\figurename{Figure}
\fi
\ifdefined\tablename
  \renewcommand*\tablename{Table}
\else
  \newcommand\tablename{Table}
\fi
}
\@ifpackageloaded{float}{}{\usepackage{float}}
\floatstyle{ruled}
\@ifundefined{c@chapter}{\newfloat{codelisting}{h}{lop}}{\newfloat{codelisting}{h}{lop}[chapter]}
\floatname{codelisting}{Listing}
\newcommand*\listoflistings{\listof{codelisting}{List of Listings}}
\makeatother
\makeatletter
\makeatother
\makeatletter
\@ifpackageloaded{caption}{}{\usepackage{caption}}
\@ifpackageloaded{subcaption}{}{\usepackage{subcaption}}
\makeatother
\usepackage{bookmark}
\IfFileExists{xurl.sty}{\usepackage{xurl}}{} % add URL line breaks if available
\urlstyle{same}
\hypersetup{
  pdftitle={UTS-4 My SHAPE --- Mineva Azzahra},
  pdfauthor={Mineva Azzahra},
  colorlinks=true,
  linkcolor={blue},
  filecolor={Maroon},
  citecolor={Blue},
  urlcolor={Blue},
  pdfcreator={LaTeX via pandoc}}


\title{UTS-4 My SHAPE --- Mineva Azzahra}
\usepackage{etoolbox}
\makeatletter
\providecommand{\subtitle}[1]{% add subtitle to \maketitle
  \apptocmd{\@title}{\par {\large #1 \par}}{}{}
}
\makeatother
\subtitle{Portfolio Asesmen II-2100 KIPP}
\author{Mineva Azzahra}
\date{2025-10-15}
\begin{document}
\maketitle

\renewcommand*\contentsname{Table of contents}
{
\hypersetup{linkcolor=}
\setcounter{tocdepth}{2}
\tableofcontents
}

\bookmarksetup{startatroot}

\chapter*{Selamat Berjumpa}\label{selamat-berjumpa}
\addcontentsline{toc}{chapter}{Selamat Berjumpa}

\markboth{Selamat Berjumpa}{Selamat Berjumpa}

\begin{figure}[H]

{\centering \includegraphics[width=9.5\linewidth,height=\textheight,keepaspectratio]{images/AZRL.png}

}

\caption{About Me}

\end{figure}%

Armein Z R Langi adalah Guru Besar di Sekolah Teknik Elektro dan
Informatika ITB, dosen ITB sejak Desember 1987, mantan Rektor
Universitas Kristen Maranatha, 1 Maret 2016 s/d 29 Februari 2020, mantan
Kepala Pusat Penelitian Teknologi Informasi dan Komunikasi (PP-TIK) ITB
November 2005 s/d Maret 2010, dan Sekretaris MWA ITB Mei 2010-Jan 2011.

Lahir di Tomohon 1962 dari pasangan Manado dan Sunda. Saat ini tinggal
di Bandung, menikah dengan Ina dan dikaruniai empat anak. Ayah dari
Gladys, Kezia, Andria, dan Marco.

Sharing pikiran singkat ada di blog \url{https://azrl.wordpress.com}.
Facebookk: armein\_langi

Andria, putri saya, lumayan sering membaca blog ini. Dia juga sumber
inspirasi tulisan saya. Dia senang menceritakan jokes pada saya, dan
kalau bagus saya tulis di sini, terutama seri Kocak Ala Andria. Dan dia
suka heran, karena isinya sudah berbeda. Memang saya senang
mengubah-ngubah cerita karena saya tidak suka menjiplak mentah-mentah.
Minggu lalu tidak sengaja dia memuji tulisan blog ini.

``Papa\ldots{} papa\ldots{} ``, tanyanya, ``Tulisan blog papa itu
copy-paste dari tulisan orang ya\ldots?''

Haa? Ya nggak mungkin lah. Kecuali lirik lagu, semua artikel di sini
ditulis sendiri.

``Ah bohong, soalnya tulisannya terlalu banyak\ldots.'' katanya tetap
tidak percaya, ``Nggak mungkinlah papa tulis sendiri\ldots{}''

Hehe, saya tersenyum sambil membelai Andria, karena buat saya itu adalah
ultimate compliment. Thanks sweetheart\ldots{}

Jadi kalau orang tidak percaya bahwa sesuatu itu karya anda, anda yang
buat, jangan marah. Itu adalah pujian yang sejati.

\bookmarksetup{startatroot}

\chapter{UTS-1 All About Me}\label{uts-1-all-about-me}

\begin{figure}[H]

{\centering \includegraphics[width=1\linewidth,height=\textheight,keepaspectratio]{images/mi.jpeg}

}

\caption{Hello}

\end{figure}%

\begin{quote}
I am both the sun and the moon, therefore, I complete myself.
\end{quote}

\begin{center}\rule{0.5\linewidth}{0.5pt}\end{center}

\section{Selamat Datang!}\label{selamat-datang}

---to Mineva's page

Saya \textbf{\emph{Mineva Azzahra}} dan saya berasal dari
\textbf{\emph{Bandung}} (warga lokal ITB). Saya berasal dari jurusan
Sistem dan Teknologi Informasi. Mahasiswa yang percaya bahwa setiap
orang adalah sistem yang unik, dan saya bersemangat untuk mempelajari
\emph{source code} yang membentuk interaksi antarmanusia. Jurusan ini
bagi saya bukan hanya tentang belajar teknologi, tetapi juga menjadi
semacam \emph{panduan pengguna} untuk memahami dunia yang semakin
kompleks di sekitar saya.

\emph{Saya menemukan bahwa efektivitas terbaik lahir dari lingkungan
yang terstruktur, namun inovasi paling signifikan justru muncul dari
kebebasan berpikir yang tidak terikat apapun!}

\section{Sedikit Perkenalan Lagi}\label{sedikit-perkenalan-lagi}

\begin{itemize}
\tightlist
\item
  \textbf{Panggilan:} Minep/Eva/Neo
\item
  \textbf{Bahasa yang Didukung:} Indonesia (Native), Inggris (Enough),
  Sunda (Native).
\item
  \textbf{Waktu Aktif:} Paling optimal di sore dan malam hari.
\end{itemize}

\section{Fokus Pengembangan Diri}\label{fokus-pengembangan-diri}

\begin{itemize}
\tightlist
\item
  \textbf{Peningkatan Keterampilan}: Meningkatkan kompetensi di bidang
  desain (khususnya UI/UX dan desain grafis), terus memperdalam keahlian
  yang relevan dengan Sistem dan Teknologi Informasi.
\end{itemize}

\bookmarksetup{startatroot}

\chapter{UTS-2 My Songs for You}\label{uts-2-my-songs-for-you}

\begin{quote}
\emph{For you, and by you---I meant, myself.}
\end{quote}

\section{Sebelah Mata}\label{sebelah-mata}

\url{https://youtu.be/u5aHyZV0CTI?si=zNwUnuZnOY9taS2E}

\begin{quote}
Tapi sebelah mataku yang lain menyadari

Gelap adalah teman setia

Dari waktu-waktu yang hilang
\end{quote}

Bukan genre lagu yang biasanya saya nikmati, tapi atmosfernya sangat
menghanyutkan. Entah mengapa ini jadi salah satu lagu yang paling sering
saya putar di tahun 2024. Jadi pengingat bahwa tidak semua hal harus
selalu ceria dan penuh semangat.

\section{In My Mouth}\label{in-my-mouth}

\url{https://youtu.be/cIi4SxZ0IC8?si=EeUQOrJCeDpnrkRK}

\begin{quote}
I don't feel like I can be anything more than this

I don't really want to be anything more than this

I just wanna be whatever you want me to be

I don't wanna have a soul
\end{quote}

Lagu ini salah satu tipe musik yang saya sukai dan ada kenangan
tersendiri. Lagunya menyajikan ekspresi emosi yang sangat jujur dan
mentah, sulit diungkapkan dengan kata-kata biasa. Perpaduan antara lirik
yang intens tentang kehancuran diri, dan keinginan akan koneksi, dengan
musik yang abrasif, menciptakan pengalaman katarsis bagi saya sendiri.

Mendengarkan ini mungkin seperti ada di pusat badai. Tapi intinya, ada
keheningan. Semua orang bisa melewati kekacauan dalam keadaan utuh.

\emph{Kamu pusatnya, bukan korbannya}.

\bookmarksetup{startatroot}

\chapter{UTS-3 My Stories for You}\label{uts-3-my-stories-for-you}

\begin{quote}
\emph{Is the story old? yes. Is it still a problem? yes.}
\end{quote}

\subsection{\texorpdfstring{\textbf{Standar Prosedur untuk Tertinggal di
Lobi}\\
}{Standar Prosedur untuk Tertinggal di Lobi }}\label{standar-prosedur-untuk-tertinggal-di-lobi}

Di dalam mobil, aku tubuh kecil yang menempel di jendela. Ada
suara-suara yang ramai dan akrab, terdengar seperti radio dari ruangan
lain, begitu dekat sekaligus jauh. Aku tidak ikut mendengarkan,
dua-duanya, aku tidak mau dan aku tidak bisa. Mataku sibuk melacak
garis-garis emas yang dibuat matahari saat menembus debu di udara, atau
menonton pohon-pohon yang nampak seperti mimpi buruk yang aku sukai. Aku
ada di antara mereka, tapi duniaku sendiri ada di balik kaca, sunyi dan
penuh dengan hal-hal kecil yang hanya aku yang melihat.

Lalu kami berhenti di sebuah tempat yang besar dan terbuka di dekat
Borobudur. Suara-suara itu menjauh, sibuk berbicara dengan orang di
balik meja kayu yang tinggi. Aku ditinggalkan sendiri untuk sesaat,
mataku terpaku pada sebuah patung dari batu di dekat kolam kecil. Aku
mau membawanya pulang, harusnya boleh.

Aku mendengar langkah kaki mereka yang menjauh, kembali ke pintu tempat
kami masuk. Salah satu dari mereka menoleh sekilas, tapi tatapannya
lewat begitu saja. Lalu sebuah pertanyaan kecil yang aneh muncul:
\emph{bagaimana kalau aku tidak ikut?} Jadi, aku tidak bergerak. Aku
hanya melangkah ke samping, bahkan tidak bersembunyi, hanya melangkah
beberapa langkah kecil. Tidak lama, aku melihat mobil itu menyala dan
perlahan pergi, membawa semua keramaian itu bersamanya.

Setelah mobil itu hilang, rasanya luar biasa? Seperti baru saja
menemukan sebuah tombol rahasia, dan dengan menekannya, suara statis
yang ramai di kepalaku, akhirnya mati. Yang tersisa bukanlah sepi,
melainkan sebuah keheningan yang jernih. Satu-satunya suara adalah
gemericik air dari kolam. Aku berjalan mendekati patung batu di kolam
itu, menyentuh kepalanya yang dingin. Tidak ada yang menelepon, tidak
ada yang panik. Untuk pertama kalinya, aku tidak hanya dipaksa mendengar
dan aku bisa memilih suara apa yang ingin kudengarkan. Dan aku memilih
ini.

⋆⋅☆⋅⋆ ──────

\textbf{\emph{Saya belajar bahwa inti dari perjalanan ini bukanlah
tentang pasrah pada alurnya, melainkan tentang mengambil tanggung jawab
penuh atas arah yang saya tuju. Masa depan mungkin terasa abstrak,
tetapi tindakan yang saya ambil hari ini sangatlah nyata. Karena itu,
kegigihan saat menghadapi kesulitan, rasa ingin tahu untuk terus
belajar, dan kemampuan beradaptasi saat rencana berubah menjadi prinsip
utama saya. Ini bukan soal menjadi sempurna, tapi soal kesadaran bahwa
sayalah yang bertanggung jawab untuk membangun jalan saya sendiri,
langkah demi langkah}}

\bookmarksetup{startatroot}

\chapter{UTS-4 My SHAPE (Spiritual Gifts, Heart, Abilities, Personality,
Experiences)}\label{uts-4-my-shape-spiritual-gifts-heart-abilities-personality-experiences}

\begin{center}\rule{0.5\linewidth}{0.5pt}\end{center}

\section{}\label{section}

\begin{quote}
\textbf{Tujuan:} Merangkum rancangan diri (charter) agar saya dapat
belajar, berkarya, dan bertumbuh secara paling selaras dengan panggilan
dan pengalaman unik saya. Dapat langsung ditempel ke halaman
\textbf{UTS-4 --- My SHAPE} dan dipakai sebagai acuan aksi 90 hari.
\end{quote}

\begin{center}\rule{0.5\linewidth}{0.5pt}\end{center}

\section{Ringkasan}\label{ringkasan}

\textbf{Peran Inti:} Pemecah masalah teknis yang berfokus pada sisi
manusia dari teknologi.

\textbf{Misi:} Mengintegrasikan analisis sistem yang logis dengan
pemahaman empati terhadap pengguna, untuk merancang dan menciptakan
solusi teknologi yang fungsional sekaligus humanis.

\textbf{Kekuatan Utama:} Pemikiran sistematis, analisis kebutuhan
pengguna, dasar-dasar desain antarmuka (UI/UX), riset mandiri, dan
kemampuan belajar yang adaptif.

\textbf{Dampak yang Dituju:} Produk digital yang intuitif, solusi
teknologi yang menjawab masalah nyata pengguna, dan kontribusi pada tim
yang menjembatani antara aspek teknis dan pengalaman pengguna.

\textbf{Peta SHAPE (singkat):}

\begin{itemize}
\tightlist
\item
  \textbf{S --- Panggilan Inti:} Analisis \& Sintesis, Empati,
  Integritas, Otonomi.
\item
  \textbf{H --- Minat \& Gairah:} Desain yang berpusat pada pengguna
  (\emph{user-centered design}); psikologi kognitif dalam interaksi
  manusia-komputer; penerapan teknologi untuk pemecahan masalah;
  mempelajari bagaimana narasi membentuk pengalaman emosional.
\item
  \textbf{A --- Abilities (Kemampuan):} Analisis \& perancangan sistem,
  dasar-dasar pemrograman (Python/Java), prototyping UI/UX (Figma),
  riset \& analisis data, komunikasi teknis tertulis.
\item
  \textbf{P --- Personality (Gaya Kerja):} Metodis \& terstruktur,
  mandiri \& proaktif (\emph{self-driven}), analitis, menghargai
  profesionalisme dan fokus pada tujuan bersama.
\item
  \textbf{E --- Experiences (Pengalaman Pembentuk):} Pengalaman formatif
  tentang pentingnya agensi dan pengambilan keputusan sadar; proyek
  akademis di STI; pembelajaran mandiri di bidang desain; analisis media
  (musik/film) untuk memahami emosi pengguna.
\end{itemize}

\begin{center}\rule{0.5\linewidth}{0.5pt}\end{center}

\section{1) S --- Panggilan Inti (Core
Calling)}\label{s-panggilan-inti-core-calling}

\begin{itemize}
\tightlist
\item
  \textbf{Analisis \& Sintesis:} Kemampuan untuk mengurai masalah
  kompleks menjadi komponen logis dan menyatukannya kembali menjadi
  sebuah solusi yang koheren.
\item
  \textbf{Empati:} Dorongan untuk memahami ``mengapa'' di balik tindakan
  pengguna; menempatkan diri pada posisi mereka untuk menemukan titik
  masalah yang sebenarnya.
\item
  \textbf{Integritas \& Otonomi:} Kebutuhan untuk bekerja secara mandiri
  dengan standar tinggi, mengambil tanggung jawab penuh atas hasil kerja
  dan proses pengambilan keputusan.
\end{itemize}

\begin{center}\rule{0.5\linewidth}{0.5pt}\end{center}

\section{2) H --- Heart (Minat Profesional \& Gairah
Intelektual)}\label{h-heart-minat-profesional-gairah-intelektual}

\begin{itemize}
\tightlist
\item
  \textbf{Desain Berpusat pada Pengguna:} Ketertarikan mendalam pada
  proses menciptakan produk yang tidak hanya berfungsi baik, tetapi juga
  mudah dan menyenangkan untuk digunakan.
\item
  \textbf{Psikologi dalam Teknologi:} Gairah untuk mempelajari bagaimana
  pikiran manusia bekerja dan menerapkan wawasan tersebut untuk
  membangun interaksi digital yang lebih baik.
\item
  \textbf{Pemecahan Masalah:} Antusiasme dalam menghadapi tantangan
  teknis atau konseptual dan merancang solusi yang efektif dan efisien.
\item
  \textbf{Analisis Naratif:} Minat dalam menganalisis bagaimana cerita,
  musik, dan media lain dapat membangkitkan emosi, sebagai dasar untuk
  merancang pengalaman yang lebih kaya.
\end{itemize}

\begin{center}\rule{0.5\linewidth}{0.5pt}\end{center}

\section{3) A --- Abilities (Kemampuan
Andal)}\label{a-abilities-kemampuan-andal}

\begin{itemize}
\tightlist
\item
  \textbf{Analisis Sistem \& Proses:} Mampu memetakan alur kerja,
  mengidentifikasi \emph{bottlenecks}, dan merancang proses yang lebih
  efisien.
\item
  \textbf{Dasar-dasar Desain UI/UX:} Menguasai prinsip-prinsip desain
  antarmuka dan pengalaman pengguna, serta mampu membuat prototipe awal
  menggunakan Figma.
\item
  \textbf{Riset \& Pembelajaran Mandiri:} Terampil dalam mencari,
  menyaring, dan mensintesis informasi baru secara mandiri untuk
  mempelajari teknologi atau metodologi baru.
\item
  \textbf{Komunikasi Tertulis:} Mampu menyusun dokumen teknis, laporan
  analisis, dan argumen desain secara jelas dan terstruktur.
\end{itemize}

\begin{center}\rule{0.5\linewidth}{0.5pt}\end{center}

\section{4) P --- Personality (Gaya Kerja
Profesional)}\label{p-personality-gaya-kerja-profesional}

\begin{itemize}
\tightlist
\item
  \textbf{Metodis \& Terstruktur:} Bekerja dengan pendekatan
  langkah-demi-langkah, memastikan setiap detail dipertimbangkan.
\item
  \textbf{Mandiri \& Proaktif:} Tidak menunggu instruksi; aktif mencari
  masalah untuk dipecahkan dan mengambil inisiatif untuk memulai.
\item
  \textbf{Fokus pada Solusi:} Ketika menghadapi masalah, orientasi utama
  adalah menemukan solusi praktis, bukan terjebak dalam diskusi yang
  tidak produktif.
\item
  \textbf{Kolaborator yang Profesional:} Menghargai komunikasi yang
  jelas, tepat sasaran, dan menjaga hubungan kerja yang saling
  menghormati.
\end{itemize}

\begin{center}\rule{0.5\linewidth}{0.5pt}\end{center}

\section{5) E --- Experiences (Pengalaman
Pembentuk)}\label{e-experiences-pengalaman-pembentuk}

\begin{itemize}
\tightlist
\item
  \textbf{Proyek Akademis STI:} Melatih kemampuan analisis, kerja tim,
  dan manajemen proyek dalam lingkungan yang terstruktur dan
  berorientasi pada hasil.
\item
  \textbf{Pengambilan Keputusan Sadar:} Pengalaman pribadi yang
  mengajarkan pentingnya agensi dan keberanian dalam mengambil keputusan
  untuk menentukan arah, bahkan saat menghadapi ketidakpastian.
\item
  \textbf{Pembelajaran Desain Mandiri:} Proses belajar UI/UX secara
  otodidak yang membangun disiplin, rasa ingin tahu, dan portofolio
  awal.
\item
  \textbf{Analisis Media:} Kebiasaan menganalisis musik, film, dan
  cerita yang mengasah kepekaan terhadap emosi dan narasi, memberikan
  wawasan unik untuk memahami pengguna.
\end{itemize}

\begin{center}\rule{0.5\linewidth}{0.5pt}\end{center}

\section{6) Piagam Diri (Self-Charter)}\label{piagam-diri-self-charter}

\textbf{Misi Profesional:} Menjadi jembatan antara dunia teknis dan
kebutuhan manusia, dengan merancang solusi teknologi yang didasari oleh
data, logika, dan empati yang mendalam.

\textbf{Nilai Inti:} Empati, Logika, Integritas, Pertumbuhan, Kualitas.

\textbf{Peran Ideal:} Analis Sistem, Desainer UX/UI, Peneliti Pengguna
(User Researcher).

\textbf{Kompas Keputusan:} (1) Apakah ini memecahkan masalah nyata bagi
pengguna? (2) Apakah solusinya logis dan efisien? (3) Apakah prosesnya
dilakukan dengan integritas? (4) Apakah ini memberikan ruang untuk
belajar dan bertumbuh?

\textbf{Janji Profesional:} Untuk selalu memulai dengan pertanyaan
``mengapa'', mendengarkan kebutuhan pengguna sebelum merancang solusi,
dan bertanggung jawab penuh atas kualitas hasil kerja.

\begin{center}\rule{0.5\linewidth}{0.5pt}\end{center}

\section{7) Narasi 90 Detik (Elevator
Pitch)}\label{narasi-90-detik-elevator-pitch}

``Saya Mineva, seorang mahasiswa Sistem dan Teknologi Informasi yang
bersemangat menjembatani antara kode dan manusia. Saya percaya teknologi
terbaik adalah yang terasa `tak terlihat' karena sangat mudah digunakan.
Dengan kemampuan

\bookmarksetup{startatroot}

\chapter{UTS-5 My Personal Reviews}\label{uts-5-my-personal-reviews}

Berikut cara saya melakukan review: mengguan chatGPT, saya mengattach
\href{skor_uts.pdf}{file promt ChatGPT}, disertai perintah :``self
assess uts-1 sanpai uts-5 dari URL
`https://ii-2100.github.io/all-about-me/'\,''

ChatGPT melakukan self-assessment UTS-1 s.d. UTS-5 langsung dari laman
yang Anda berikan dan menilai memakai rubrik tugas UTS (skala 1--5 per
kriteria). Rekap skor siap diunduh sebagai CSV:
\href{sandbox:/mnt/data/UTS_self_assessment.csv}{Download CSV
ringkasan}.

\bookmarksetup{startatroot}

\chapter{Hasil Self-Assessment UTS (URL:
ii-2100.github.io/all-about-me)}\label{hasil-self-assessment-uts-url-ii-2100.github.ioall-about-me}

\section{Identifikasi}\label{identifikasi}

\begin{itemize}
\tightlist
\item
  Nama \& NIM penulis: \textbf{Armein Z. R. Langi -- 131902360} (tertera
  di halaman depan portofolio).
  (\href{https://ii-2100.github.io/all-about-me/}{II 2100})
\item
  Penilai: \textbf{Self-assessment (Armein Z. R. Langi)}
\item
  Catatan cakupan: halaman beranda memuat ``About Me''; navigasi ke ``My
  Songs for You'', ``My Stories for You'', ``My Shapes'', dan ``My
  Personal Reviews'' tersedia.
  (\href{https://ii-2100.github.io/all-about-me/}{II 2100})
\end{itemize}

\section{Tinjauan Umum}\label{tinjauan-umum}

\begin{itemize}
\tightlist
\item
  \textbf{UTS-1 (All About Me)} hadir di beranda (``Selamat Berjumpa /
  About Me''). Isi memperkenalkan identitas dan latar personal secara
  padat. (\href{https://ii-2100.github.io/all-about-me/}{II 2100})
\item
  \textbf{UTS-2 (My Songs for You)} memuat judul karya dan tautan audio,
  namun lirik/isi tidak ditampilkan di halaman (file audio tidak bisa
  saya akses dari sini), sehingga penilaian konten terbatas pada
  kelengkapan presentasi.
  (\href{https://ii-2100.github.io/all-about-me/My_Song_for_You/index.html}{II
  2100})
\item
  \textbf{UTS-3 (My Stories for You)} berisi tautan ke beberapa cerita;
  konten naratif kuat, terutama pada kisah keluarga ``Gaun Pengantin
  Gladys'' serta kisah inspiratif lainnya.
  (\href{https://ii-2100.github.io/all-about-me/My_Stories_for_You/index.html}{II
  2100})
\item
  \textbf{UTS-4 (My SHAPE)} saat ini halaman kosong/tanpa isi, hanya
  navigasi.
  (\href{https://ii-2100.github.io/all-about-me/My_Shapes/index.html}{II
  2100})
\item
  \textbf{UTS-5 (My Personal Reviews)} berisi metode/tautan panduan
  review, namun belum ada contoh \textbf{review personal} yang lengkap
  terhadap sebuah pesan/teks.
  (\href{https://ii-2100.github.io/all-about-me/My_Personal_Reviews/index.html}{II
  2100})
\end{itemize}

\begin{center}\rule{0.5\linewidth}{0.5pt}\end{center}

\section{Tinjauan Spesifik + Skor
(1--5)}\label{tinjauan-spesifik-skor-15}

\subsection{UTS-1 --- All About Me (di
beranda)}\label{uts-1-all-about-me-di-beranda}

\textbf{Skor per kriteria:} Orisinalitas \textbf{3}, Keterlibatan
\textbf{3}, Humor \textbf{1}, Wawasan/Insight \textbf{3} → \textbf{Total
10/20 (50\%)}. \textbf{Alasan singkat:} Perkenalan diri faktual dan
jelas, namun gaya masih informatif ketimbang naratif (minim humor dan
moment ``hook''). (\href{https://ii-2100.github.io/all-about-me/}{II
2100}) \textbf{Saran perbaikan:} Buka dengan anekdot pribadi (1--2
paragraf) yang ``mengikat'' (mis. titik balik karier/keluarga),
tambahkan satu momen humor ringan, lalu tutup dengan refleksi singkat
tentang nilai/visi diri agar aspek insight naik.

\subsection{UTS-2 --- My Songs for You}\label{uts-2-my-songs-for-you-1}

\textbf{Skor per kriteria:} Orisinalitas \textbf{2}, Keterlibatan
\textbf{2}, Humor \textbf{1}, Inspirasi \textbf{2} → \textbf{Total 7/20
(35\%)}. \textbf{Alasan singkat:} Halaman menampilkan judul lagu ``River
in my Mind'', ``Heaven on Earth'', namun tanpa lirik/cerita di balik
lagu sehingga sulit menilai aspek pesan, humor, dan inspirasi.
(\href{https://ii-2100.github.io/all-about-me/My_Song_for_You/index.html}{II
2100}) \textbf{Saran perbaikan:} Tambahkan lirik lengkap, 1 paragraf
cerita proses kreatif, dan 2--3 kalimat ``pesan untukmu'' agar inspirasi
terbaca; sertakan player/tautan yang dapat diputar langsung + fallback
transkrip.

\subsection{UTS-3 --- My Stories for
You}\label{uts-3-my-stories-for-you-1}

\textbf{Skor per kriteria:} Orisinalitas \textbf{5}, Keterlibatan
\textbf{5}, Pengembangan Narasi \textbf{4}, Inspirasi \textbf{5} →
\textbf{Total 19/20 (95\%)}. \textbf{Alasan singkat:} Cerita ``Gaun
Pengantin Gladys'' dkk sangat personal, emosional, dan inspiratif; ritme
bertutur hidup serta detail situasional kuat (konten ramu unsur
penebusan/keluarga/iman).
(\href{https://ii-2100.github.io/all-about-me/My_Stories_for_You/index.html}{II
2100}) \textbf{Saran perbaikan:} Tambah ``lead'' 2--3 kalimat yang
merangkum pesan kunci tiap cerita; akhiri dengan ajakan/refleksi 1--2
kalimat agar resonansi ke pembaca makin jelas.

\subsection{UTS-4 --- My SHAPE}\label{uts-4-my-shape}

\textbf{Skor per kriteria:} Orisinalitas \textbf{1}, Keterlibatan
\textbf{1}, Pengembangan \textbf{1}, Inspirasi \textbf{1} →
\textbf{Total 4/20 (20\%)}. \textbf{Alasan singkat:} Konten belum
tersedia.
(\href{https://ii-2100.github.io/all-about-me/My_Shapes/index.html}{II
2100}) \textbf{Saran perbaikan (prioritas):}

\begin{enumerate}
\def\labelenumi{\arabic{enumi}.}
\tightlist
\item
  Tampilkan rekap SHAPE (Spiritual gifts, Heart, Abilities, Personality,
  Experiences) dalam 1 tabel ringkas + 3--5 butir bukti;
\item
  Narasikan 2 pengalaman kunci (±150 kata/each) yang menunjukkan
  ``signature strengths'';
\item
  Tutup dengan 3 rencana aksi konkret (who/what/when).
\end{enumerate}

\subsection{UTS-5 --- My Personal
Reviews}\label{uts-5-my-personal-reviews-1}

\textbf{Skor per kriteria:} Pemahaman Konsep \textbf{2}, Analisis Kritis
\textbf{1}, Argumentasi (Logos) \textbf{1}, Etos \& Empati \textbf{2},
Rekomendasi \textbf{1} → \textbf{Total 7/25 (28\%)}. \textbf{Alasan
singkat:} Halaman berisi metode cara menilai, tetapi \textbf{belum ada}
contoh \textbf{review personal} yang lengkap terhadap sebuah pesan/teks
sehingga aspek analisis-argumentasi tak terbaca.
(\href{https://ii-2100.github.io/all-about-me/My_Personal_Reviews/index.html}{II
2100}) \textbf{Saran perbaikan:} Pilih 1 karya personal (mis.
UTS-1/2/3), tulis review 400--600 kata: ringkas pesan, nilai dengan
rubrik, berikan 2--3 bukti kutipan, evaluasi etos/empati, lalu tutup
dengan rekomendasi perbaikan spesifik.

\begin{center}\rule{0.5\linewidth}{0.5pt}\end{center}

\section{Rekap Skor (ringkas)}\label{rekap-skor-ringkas}

\begin{itemize}
\tightlist
\item
  \textbf{UTS-1:} 10/20 → \textbf{50\%}
\item
  \textbf{UTS-2:} 7/20 → \textbf{35\%}
\item
  \textbf{UTS-3:} 19/20 → \textbf{95\%}
\item
  \textbf{UTS-4:} 4/20 → \textbf{20\%}
\item
  \textbf{UTS-5:} 7/25 → \textbf{28\%}
\end{itemize}

CSV lengkap sudah saya siapkan untuk dokumentasi dan olah lanjut:
\href{sandbox:/mnt/data/UTS_self_assessment.csv}{Download CSV
ringkasan}.

\section{Langkah Perbaikan Cepat (prioritas 1
minggu)}\label{langkah-perbaikan-cepat-prioritas-1-minggu}

\begin{enumerate}
\def\labelenumi{\arabic{enumi}.}
\tightlist
\item
  \textbf{Lengkapi UTS-4 (My SHAPE)} sesuai saran di atas;
\item
  \textbf{Tambahkan lirik + sinopsis} untuk tiap lagu di UTS-2 agar bisa
  dinilai isi/inspirasinya;
\item
  \textbf{Tulis 1 review personal penuh} (UTS-5) memakai rubrik (pilih
  salah satu karya Anda sendiri sebagai objek);
\item
  Revisi UTS-1 dengan pembuka anekdot + refleksi nilai diri.
\end{enumerate}

Jika Anda mau, saya bisa bantu merapikan UTS-4 (tabel SHAPE + narasi)
dan membuat kerangka cepat untuk \textbf{review} di UTS-5 dari salah
satu karya Anda.

\bookmarksetup{startatroot}

\chapter{UAS-1 My Concepts}\label{uas-1-my-concepts}

Mau hidup epik ? \href{lifestory.pdf}{Write your Life Story}

Apa itu berkonsep?

\url{https://youtu.be/QVfUlVBO80U?si=yM6q_rwV9rcDBbu7}

\bookmarksetup{startatroot}

\chapter{UAS-3 My Opinions}\label{uas-3-my-opinions}

SApa itu beropini? \href{BM\%20Opini.mp4}{Opini Berpengaruh}

Bagiamana menjaadi menarik? \href{./Interesting.mp4}{Menjadi Menarik}

\bookmarksetup{startatroot}

\chapter{UAS-3 My Innovations}\label{uas-3-my-innovations}

\bookmarksetup{startatroot}

\chapter{UAS-4 My Knowledge}\label{uas-4-my-knowledge}

Cara saya mengkomunikasikan sebuah pengatahuan sebagai petunjuk bagi
orang lain 1) saya tulis
\href{Rekomendasi\%20Presentasi\%20Efektif(Contoh\%20Makalah).pdf}{makalah
sebagai bahan utama} 2) lalu saya buat
\href{Contoh\%20Transkrip\%20Presentasi.pdf}{transkrip ucapan lisan} 3)
kemudian saya kembangkan
\href{Rekomendasi\%20Presentasi\%20(Contoh\%20Slides).pdf}{slide
pendukung trnsskrip} 4) lalu saya memproduksivideo audio visual
\url{https://youtu.be/ZbghfMvnPZc} \url{https://youtu.be/ZbghfMvnPZc}

\bookmarksetup{startatroot}

\chapter{UAS-5 My Professional
Reviews}\label{uas-5-my-professional-reviews}

Untuk melAkukan review, seperti pada
\href{../My_Personal_Reviews/Doc.5.Mengevaluasi-Esai-Berdasarkan-Rubrik.pdf}{pendekatan
AI}, kita membutuhkan rubrik

\bookmarksetup{startatroot}

\chapter{Summary}\label{summary}

In summary, this book has no content whatsoever.

\bookmarksetup{startatroot}

\chapter*{References}\label{references}
\addcontentsline{toc}{chapter}{References}

\markboth{References}{References}

\phantomsection\label{refs}




\end{document}
